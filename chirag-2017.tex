\documentclass[12pt,a4paper,sans]{moderncv}
\moderncvstyle{casual}
\moderncvcolor{grey}
\nopagenumbers{}
\usepackage[scale=0.80]{geometry}

\name{Chirag}{Kantharia}
\address{201, Rahat Residency, Nagavarapalya-Varthur Road, C V Ramanagar, Bangalore, INDIA}
\phone{+91.98455.68691}
\email{chirag.kantharia@gmail.com}
\social[linkedin]{https://in.linkedin.com/in/chirag-kantharia}
\social[github]{https://github.com/chyrag}
% \extrainfo{* 01 December 1976}

\begin{document}
\makecvtitle

\section{Summary}
\cvlistitem{Experienced system software developer on Linux and FreeBSD with over
  17 years of experience, passionate about building networking products that
  sustain test of time, with proclivity for minimalism, beautiful code and
  elegant architecture. Identifying bottlenecks, building productivity tools,
  automating work flows are my areas of interest.}

\section{Strengths}
\cvlistitem{Platform Software, Architecture, Networking, Virtualization}
\cvlistitem{Distributed computing, Clusters, Embedded systems}
\cvlistitem{C, BASH, make, GDB, Git, SVN, CVS}

\section{Experience}

\cventry{Jan 2015 - Dec 2016}{Software Engineer}{Versa Networks Inc}{Bangalore}{India}{}{}

\cvlistitem{I developed and maintained publisher-subscriber model based
  infrastructure for control plane and data plane redundancy for an SDN
  controller intra-chassis high availability setup. This helped the standby node
  to gracefully take over the mastership of the high availability setup and
  ensure the SDN controller continues to route the traffic when the active node
  becomes unavailable.}

\vspace{10px}
\cventry{Nov 2004 - Dec 2014}{Staff Engineer}{Juniper Networks Inc}{Bangalore}{India}{}{}

\cvlistitem{I developed publisher-subscriber based server backend for an SDK for
  programmable networks. The backend enabled Juniper routers to push network
  events to control plane software running on an external box which could then
  take action on the event.}

\cvlistitem{I co-delivered hardened, high-performance service provider
  appliance, CSE2000, that acted as an external service card on PTX5000, which
  delivered wide range of controller, scaling and network function
  virtualization (NFV) applications. My responsibility was the control plane
  software including chassis management which comprised of reading sensor data
  off I2C devices, managing alarm conditions, handling IPMI events, pushing
  interface events to the main control board running JUNOS, thereby, enabling
  the appliance to appear as a plugged in field replaceable unit (FRU) to the
  the management software running on main control board.}

\cvlistitem{I developed control plane software for T4000 line card, including
  board initialization, communciation with the control board running JUNOS, and
  remote upgrade procedure for the FPGA on the board.}

\cvlistitem{I co-developed infrastructure for in-service software upgrade for
  Juniper T-series, M-series and TX-series routers. My responsibility comprised
  of the in-service upgrade control flow, and replication of kernel state across
  heterogenous kernel versions. The feature enables Juniper routers to carry out
  software upgrade in production environment with minimal or zero traffic loss.}

\cvlistitem{I co-developed GRES (Graceful RE Switchover) infrastructure for high
  availability and redundancy for control plane objects, which comprised of
  replicating kernel state onto the standby node on Juniper routers with dual
  routing engines.}

\cvlistitem{I developed tools for monitoring scaling numbers for Juniper routers
  across releases. This helps the competitive edge team to ensure that Juniper
  routers scaled to the published numbers.}

\cvlistitem{I created a HAL abstraction layer making it easier to port JUNOS to
  new platforms. This enabled the porting of JUNOS effort equivalent of writing
  a device driver for the new platform. The modularization was flexible to allow
  addition of new platform specific routines if required. I, also, developed
  diagnostic software to validate new hardware including NICs and ethernet
  switches on the Juniper platforms.}

\cvlistitem{Feature enhancement and sustenance of JUNOS kernel.}

\vspace{10px}
\cventry{Jun 2003 - Oct 2004}{Senior Software Engineer}{Hewlett-Packard}{Bangalore}{India}{}{}

\cvlistitem{I developed parallel boot mechanism for OpenSSI, Linux cluster
  project sponsored by HP, which helped faster bootup of the significantly large
  Linux cluster. I was also official release engineer for the project for
  several releases. I also helped beta customers to scale their applications on
  the cluster. I also helped port Linux Test Project to OpenSSI to help weed out
  bugs related to system call behaviour on the cluster infrastructure.}

\cvlistitem{I maintained CPQARRAY, CPQFC device drivers in the mainline Linux
  kernel, and support for the drivers in HP volume manager utility. I created
  the distribution packages for various mainstream Linux distributions including
  RedHat and SUSE.}

\vspace{10px}
\cventry{Mar 2002 - Apr 2003}{Software Engineer}{Timesys Inc}{Bangalore}{India}{}{}

\cvlistitem{I ported Timesys Linux to various PowerPC and StrongARM based single
  board computers from Embedded Planet and Applied Data Systems, which comprised
  of getting the Timesys Linux to boot up on the board and getting the serial
  console and ethernet working. I also ported framebuffer device driver,
  microwindows, picogui on few single board computers with LCD display.}

\vspace{10px}
\cventry{Dec 1999 - Feb 2002}{Software Engineer}{Epigon Audiocare}{Bangalore}{India}{}{}

\cvlistitem{I developed a clean-room implementation of Java Virtual Machine
  adhering to J2ME/MIDP specifications intended to run on a Linux 2.2 based
  embedded system. The target system was a Netsilicon manufactured Net+Lx single
  board computer.}

\cvlistitem{Developed Linux 2.4 device driver for in-house developed PCI card
  for C-sound music synthesis. The device processed the C-sound input and
  generated MIDI files which could be downloaded by the user upon being notified
  by the driver.}

\section{Education}
\cventry{1995--2000}{B.Tech}{Indian Institute of Technology, Bombay}
        {Powai}{Mumbai} {Computer Science and Engineering}


\end{document}
